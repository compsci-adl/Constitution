\documentclass{constitution}

\setAdoptedLongDate{13th of March, 2024}

\begin{document}

\begin{titlepage}
    \begin{center}
        \vspace{1cm}
        \LARGE\textbf{\textit{The}} \\
        \huge\textbf{CONSTITUTION} \\
        \LARGE\textbf{\textit{of}} \\
        \huge\textbf{THE UNIVERSITY OF ADELAIDE COMPUTER SCIENCE CLUB}

        \vfill
        \includesvg[width=0.7\textwidth]{img/CSLogo.svg}
        \vfill

        \Large
        Adopted at the Annual General Meeting on the \adoptedLongDate.
    \end{center}
\end{titlepage}


\pagenumbering{roman}
\tableofcontents
\newpage

\pagenumbering{arabic}


\section{Principles and objectives of the Club}\label{principlesObjectivesClub}

\subsection{Primary objectives}\label{primaryObjectivesClub}
The primary objective of the Club is to provide collaboration, networking, intellectual stimulation, and social opportunities, as well as any other pursuit reasonably suited for students studying computer science at the University of Adelaide and other interested people.

\subsection{Not for profit clause}\label{notForProfitClause}
The assets and income of the Club shall be applied solely in furtherance of its above-mentioned objects and no portion shall be distributed directly or indirectly to the members of the Club except as bona fide compensation for services rendered or expenses incurred on behalf of the Club.

\subsection{Status of Constitution}\label{statusConstitution}
All rules, regulations, requirements and steps outlined by the YouX and the Clubs Policy take precedence over anything outlined in this constitution.

\section{Name}\label{nameClub}
\begin{enumerate}[(1)]
    \item The legal and formal name of the club is \textbf{The University of Adelaide Computer Science Club}.
    \item For the purposes of publicity, publication, or other purposes as approved by the President or their delegate, the name of the Club may be written as ``CS Club''. It may also be written as a syntactically valid source code statement or equivalent for any self-hosting computer programming language which calls a zero-parameter callable unit named ``club'' contained or referenced within a first-class object named ``cs'', discarding any returned value. These names need not be capitalised as described.
\end{enumerate}


\section{Definitions}\label{definitions}
Unless a contrary intention is evident, the following definitions apply to this Constitution and all other club documents:
\begin{description}
    \item \defn{Academic day} means a weekday defined by the University of Adelaide to be within a semester;
    \item \defn{AGM} means an Annual General Meeting of the Club;
    \item \defn{Club} means The University of Adelaide Computer Science Club;
    \item \defn{Committee} means the Committee of the Club;
    \item \defn{Constitution} means this constitution;
    \item \defn{Executive} means the executive body as defined in \Cref{executiveBody};
    \item \defn{GM} means a General Meeting of the Club;
    \item \defn{Office} means an elected position in the Club provided for in \Cref{committeeComp};
    \item \defn{Officer} means a member who has been elected to an office;
    \item \defn{SGM} means a Special General Meeting of the Club;
    \item \defn{University} means the University of Adelaide;
    \item \defn{YouX} means the student union of the University.
\end{description}


\section{Membership}\label{membershipClub}

\subsection{General membership}\label{generalMembership}
General membership in the Club is open to any current undergraduate or postgraduate student at the University of Adelaide who:
\begin{enumerate}[(a)]
    \item pays the membership fee as determined under \Cref{membershipFee}; and
    \item submits to the committee the following information:
          \begin{enumerate}[(i)]
              \item full name; and
              \item student ID number; and
              \item preferred email address.
          \end{enumerate}
\end{enumerate}

\subsection{Associate membership}\label{associateMembership}
\begin{enumerate}[(1)]
    \item Associate membership in the Club is open to any person not a student of the University of Adelaide only if the applicant may satisfy any possible reasonable objections the committee may raise.
    \item Associate members must also satisfy the criteria under \Cref{generalMembership}.
\end{enumerate}

\subsection{Honorary membership}\label{honoraryMembership}
Honorary membership in the Club is open to any person upon whom the Club has conferred. Such a membership may only be conferred by a resolution at a GM.

\subsection{Refusal and revocation of membership}\label{refusalRevocationMembership}
\begin{enumerate}[(1)]
    \item The committee has the right to refuse or revoke the membership of any potential or current member if:
          \begin{enumerate}[(a)]
              \item the applicant or member has proven detrimental to the interests of the club; or
              \item violates the Code of Conduct of the University of Adelaide; or
              \item is otherwise causing serious issues within the club for its members and/or committee.
          \end{enumerate}
    \item Membership may only be refused or revoked after a passing motion of a two-thirds majority during a committee meeting. The person to whom membership is being refused or revoked will be given an opportunity to attend the meeting and present their case to the committee, whether in person, electronically, or in writing. The committee will make reasonable efforts to hold the meeting during a time in which the person may attend to plead their case.
    \item In the event that membership is being refused or revoked due to issues arising out of interpersonal conflict, the other parties seeking the refusal or revocation will also be permitted to attend the committee meeting and present their arguments.
\end{enumerate}

\subsection{Membership fee}\label{membershipFee}
\begin{enumerate}[(1)]
    \item The club's membership fee is determined at each AGM for the following year.
    \item The membership fee provides membership until the 1st of January in the following year.
    \item Honorary members of the club are exempt from membership fees as set out in \Cref{honoraryMembership}.
\end{enumerate}

\subsection{Termination of membership}\label{terminationMembership}
\subsubsection{Ordinary membership termination}\label{ordinaryMembershipTerimnation}
The committee may suspend any membership for whatever reason if:
\begin{enumerate}[(a)]
    \item the member is given a written notice of suspension; and
    \item a Committee Meeting is held to vote on the revocation within 3 weeks of suspension; and
    \item the member is notified in writing within 48 hours of the conclusion of the committee meeting.
\end{enumerate}

\subsubsection{Suspended members}\label{suspendedMembers}
During the process as described in \Cref{ordinaryMembershipTerimnation}:
\begin{enumerate}[(a)]
    \item suspended members may not attend club events or otherwise take part in club activities until a resolution is reached; and
    \item if the committee meetings result in the revocation of a member's membership, their membership is terminated immediately.
\end{enumerate}

\subsubsection{Member's own choice}
Members of the club may terminate their own membership by writing to the committee.

\subsubsection{Fee refund}
Members who have their membership terminated, whether by their own accord or through a decision of the committee, are no longer affiliated with the club and cannot have their membership fee refunded.


\section{The Committee}\label{theCommittee}

\subsection{Composition}\label{committeeComp}
\begin{enumerate}[(1)]
    \item The composition of the committee is to be laid out in Schedule 1 to the Constitution, while the composition and responsibilities of the Executive body is defined in \Cref{committeeComp}.
    \item In order of precedence, the committee members are the President, the Vice-President, the Treasurer, the Secretary, and the Partnerships \& Sponsorships Manager, who form the Executive Committee.
    \item The order of precedence for the remaining committee members is to be laid out in Schedule 1 to the Constitution, in accordance with the rules outlined in \Cref{schedule1}.
\end{enumerate}

\subsection{Election}
\begin{enumerate}[(1)]
    \item A committee is to be elected to office at the AGM, with any exceptions for non-executive positions to be laid out in Schedule 1.
    \item A turnover meeting must be held within one month of any general meeting where a new committee member is elected to train the new committee member(s). This time frame may be extended if it clashes with exams or major assessment deadlines.
\end{enumerate}

\subsection{Duties}
The committee acts on behalf of the Club in the general day-to-day running of the Club and must not act to the detriment of the Club's interests.

\subsection{Casual vacancies}\label{committeeCasualVacancies}
The committee has the power to fill any casual vacancies arising from circumstances outlined in \Cref{extraordinaryCommitteeChanges} during its term of office. All members shall be given reasonable ability, time and opportunity to apply for casual vacancies. The ability of the committee to fill casual vacancies shall not be restricted by this constitution.

\subsection{Expiration}
All offices expire at the AGM, but former officers may re-run and hold office any number of times.

\subsection{Committee meetings}
The committee must meet at least once a month during the academic year. The committee will make reasonable efforts to ensure maximal attendance by committee members, including but not limited to:
\begin{enumerate}[(a)]
    \item provisions to attend meetings electronically; or
    \item partial attendance in order to raise their proposed motions; or
    \item sending in a representative upon written notice to and approval by the President and Secretary; or
    \item sending in a written summary of their motions, reports, and voting desires for other motions and positions; or
    \item polls by the President, whether over Messenger or on voting websites, including but not limited to WhenIsGood and Doodle, to permit committee meetings to be held during times of maximum availability. % PROPOSAL: Make this more general
\end{enumerate}

\subsection{Removal of committee members due to absences}\label{committeeRemovalAbsences}
\begin{enumerate}[(1)]
    \item If a committee member fails to attend three committee meetings in a row without providing an apology, being on a leave of absence, or having alternative arrangements put in place, the committee member will be immediately removed from their position.
    \item The rules for resetting the count are as follows:
          \begin{enumerate}[(a)]
              \item An attendance will reset the count.
              \item Apologies will not be counted towards the three absences count.
              \item Apologies will not reset the count.
          \end{enumerate}
\end{enumerate}

\subsection{Removal of committee members due to misconduct}
\begin{enumerate}[(1)]
    \item If a committee member has demonstrable interests against the club, or actively works to undermine the club and its interests, they may be removed from their positions subject to a vote of no less than two-thirds of the committee.
    \item If a committee member is accused of harassment against another committee member, they will be suspended from their position and may be removed after a committee meeting subject to a vote of no less than two-thirds of the committee.
          \begin{enumerate}[(a)]
              \item The accused will not be allowed to attend the committee meeting in person should the victim be present, to allow the victim to safely attend the meeting.
              \item The accused will be allowed to attend electronically and will be allowed to submit a written statement.
          \end{enumerate}
    \item If a committee member is convicted of violating a University Code of Conduct resulting in their suspension from the University, they will be immediately removed from their position as a committee member.
    \item If a committee member commits a crime resulting in jail time, they will be immediately removed from their position as a committee member.
    \item If a committee member is found to be committing fraudulent activities using the club bank account, they will be immediately suspended from their position as a committee member, and a meeting with YouX will be held to discuss the offence.
          \begin{enumerate}[(a)]
              \item If the funds are not returned, the Club will terminate both their position and their membership.
              \item The committee will pursue all reasonable pathways to have the funds returned, and may seek legal advice in the event that other pathways have been exhausted.
          \end{enumerate}
    \item If a committee member is continually absent from meetings throughout the year, is consistently failing to fulfil their duties or tasks, is causing conflict within the committee or club, or is otherwise proven to be difficult to work with, they may be removed from their position at a committee meeting subject to a two-thirds majority vote.
\end{enumerate}

\subsection{Eligibility to hold office}
\begin{enumerate}[(1)]
    \item Committee members must fulfil all eligibility requirements in order to hold office.
          \begin{enumerate}[(a)]
              \item Committee members must be current members of the Club.
              \item The committee reserves the right to prohibit the election of a nominee that has been previously removed from the committee for reasons of misconduct, as outlined in Section 5.8(2), 5.8(3), 5.8(4), and 5.8(5), by a vote of no less than two-thirds of the committee.
          \end{enumerate}
    \item Eligibility to hold Executive positions is outlined in \Cref{execEligibility}.
\end{enumerate}


\section{Special resolutions of the Committee}\label{specialResolutionsCommittee}

\subsection{Definition}
A special resolution is a motion put forth by the committee on a specific matter which is \textbf{not}:
\begin{enumerate}[(a)]
    \item related to the day-to-day running of the club; or
    \item already accounted for by constitutional or executive powers.
\end{enumerate}

\subsection{Process}
A special resolution must be raised during a committee meeting as set out in \Cref{committeeMeetings} and voted upon as in \Cref{committeeMeetingVotingProcedure}. A special resolution is required for a matter to be brought to a GM to be decided.
For the resolution to take effect, the resolution must be put to vote in a GM as set out in \Cref{generalMeetings}.

\section{Committee meetings}\label{committeeMeetings}

\subsection{Calling a meeting}
A meeting is called by the President or the Vice-President to discuss the day-to-day running of the club and any issues for the committee to decide. The meeting must be called at least 1 week in advance; if they are not, committee members may be absent without reason and suffer no consequence as set out in \Cref{committeeRemovalAbsences}.

\subsection{Voting procedure}\label{committeeMeetingVotingProcedure}
\begin{enumerate}[(1)]
    \item The President has a deliberative vote and in the case of an equality of votes, may exercise a casting vote. This means they effectively have 2 votes, 1 in the capacity of a member and another in the capacity of the chair of the meeting.
    \item The President chairs all committee meetings and GMs of the Club. In their absence, the chair of the meeting will follow in the order of precedence stated in \Cref{committeeComp}.
    \item No committee member may delegate his/her voting right in absentia.
    \item The quorum of a committee meeting is 5 distinct voting members of the committee.
    \item If a quorum is not present then the executive has general decisive power.
\end{enumerate}

\subsection{Minutes}
Minutes of all meetings must be kept and they must include:
\begin{enumerate}[(a)]
    \item the names of the persons that attended that meeting; and
    \item discussion points; and
    \item decisions of the committee.
\end{enumerate}

\subsection{Special circumstances}
Upon receiving a written petition of 3 voting committee members, the President, or in their absence the Vice-President, must call a meeting of the committee within 14 days.

\section{Extraordinary Committee changes}\label{extraordinaryCommitteeChanges}
\subsection{Cessation of office}
A member of the committee ceases holding their office if:
\begin{enumerate}[(a)]
    \item the President or Vice-President receives a written notice of resignation from that member; or
    \item the member is absent, without leave of absence being granted by resolution of the committee, for 3 consecutive meetings of the committee of which the member was notified; or
    \item their membership in the club is terminated as set out in \Cref{terminationMembership}; or
    \item a motion of no-confidence in a committee member's ability to perform their duties is expressed by at least 6 out of 9 possible members, whereby an SGM must be called to vote on the matter of changing the committee. % PROPOSAL: Change to two-thirds majority
\end{enumerate}

\subsection{Filling a vacancy}
\begin{enumerate}[(1)]
    \item When an office becomes vacant, a GM must called to fill the position within 6 weeks unless:
          \begin{enumerate}[(a)]
              \item the AGM is to be held within 8 weeks of the position becoming vacant; or
              \item there is no academic day within the next 6 weeks. In this instance, the GM is to be held within 2 weeks of the next academic day; or
              \item the office is filled as a casual vacancy as specified in \Cref{committeeCasualVacancies}.
          \end{enumerate}
    \item The President, or in their absence, the Vice-President, may appoint any volunteering club member to fill the office in the interim.
\end{enumerate}

\section{The Executive Body}\label{executiveBody}

\subsection{Composition}
The Executive consists of:
\begin{enumerate}[(a)]
    \item the President; and
    \item the Vice-President; and
    \item the Treasurer; and
    \item the Secretary; and
    \item the Partnerships \& Sponsorships Manager.
\end{enumerate}

\subsection{General power}
The Executive has general power to make regulations necessary to put into effect this constitution, provided that such regulations are consistent with this constitution and the objectives of the club.

\subsection{Maintaining a Committee Composition Schedule}\label{schedule1}
\begin{enumerate}[(1)]
    \item The Executive shall be responsible for maintaining Schedule 1, titled ``Committee Composition''.
    \item The Schedule shall detail the composition of non-executive offices in the committee, and must include:
          \begin{enumerate}[(a)]
              \item the positions that exist and the number of positions available; and
              \item the eligibility requirements for each position; and
              \item the duties for each position; and
              \item the order of precedence for each position.
          \end{enumerate}
    \item The Schedule may contain additional details about the composition, but not in replacement to the aforementioned clauses.
    \item The Schedule may contain information about the executive committee, pertaining to their duties and eligibility criteria. However, this does not replace the rules as outlined in this constitution.
    \item Minor typographical and formatting changes may be made to the Schedule without the need for a resolution of the committee.
\end{enumerate}


\subsection{Acting presidencies}
In the event the President is unable to fulfil their duties, the executive has the power to promote, in the order of precedence set out in \Cref{committeeComp}, a member of the executive to acting President subject to ratification at GM.

\subsection{Eligibility to hold positions}\label{execEligibility}
\begin{enumerate}[(1)]
    \item Those serving in an executive office must have previously served in another position on the committee. An ordinary member of the club may only hold this office where no other valid nomination has been received.
    \item The President and Treasurer must be a student of the University of Adelaide at the time of their appointment and through the entirety of their term of office.
\end{enumerate}

\subsection{Working groups}
The executive has the power to appoint working groups from within the membership to perform duties associated with a specific agenda. Such working groups are, at all times, answerable to the executive.

\subsection{Duties}

\subsubsection{The President}
The President is responsible for:
\begin{enumerate}[(a)]
    \item presiding over all meetings of the Club, the Committee, and the Executive; and
    \item representing the Club in all official matters; and
    \item ensuring that the Club is run in accordance with the Constitution and its Schedules; and
    \item ensuring that the Club is run in accordance with the Clubs Policy of YouX;
          and
    \item calling meetings of the Committee, the Executive, and General Meetings;
          and
    \item the overall management of the Club.
\end{enumerate}

\subsubsection{The Vice-President}
The Vice-President is responsible for:
\begin{enumerate}[(a)]
    \item assisting the President in the overall management of the Club; and
    \item acting as President in the absence of the President.
\end{enumerate}

\subsubsection{The Treasurer}
The Treasurer is responsible for:
\begin{enumerate}[(a)]
    \item the financial management of the Club; and
    \item the safekeeping of the Club's funds; and
    \item ensuring that the Club's financial records are kept up to date; and
    \item delivering a financial report to the Annual General Meeting; and
    \item delivering a financial report to the Executive at the end of the first semester.
\end{enumerate}

\subsubsection{The Secretary}
The Secretary is responsible for:
\begin{enumerate}[(a)]
    \item the administrative management of the Club; and
    \item the safekeeping of the Club's records; and
    \item minuting all meetings of the Club, the Committee, and the Executive; and
    \item minuting all General Meetings; and
    \item ensuring that the Club's records are kept up to date.
\end{enumerate}

\subsubsection{The Partnerships \& Sponsorships Manager}
The Partnerships \& Sponsorships Manager is responsible for:
\begin{enumerate}[(a)]
    \item the management of the Club's partnerships and sponsorships; and
    \item the establishment of new partnerships and sponsorships; and
    \item the maintenance of existing partnerships and sponsorships; and
    \item delivering a report to the Annual General Meeting.
\end{enumerate}

\section{Club finances}

\subsection{Bank account}
The club must hold a bank account for the purpose of holding the club's funds.

\subsection{Signatories}
The signatories of the club’s bank account are the President and Treasurer.

\subsection{Minimum balance}
The committee shall endeavour to maintain a minimum bank balance of \$2,000, except where there is an essential need for the spending of this money to run events and essential functions as approved in a committee meeting.

\subsection{Authority to access funds}
\begin{enumerate}[(1)]
    \item Any one signatory may withdraw funds from the bank account with written approval from the rest of the Executive by any means of communication, such communication must be recorded in the club's records for auditing purposes.
    \item The use of these funds must then be approved by a majority of no less than half of the committee at the subsequent committee meeting or the funds must be returned immediately.
    \item All members of the Executive will have the authority to view the bank account and challenge unauthorised transactions.
    \item Should the Executive fail to unanimously authorise the withdrawal of funds, a committee meeting must be held before funds can be withdrawn. Upon a successful motion of no less than half of the committee, the funds may then be withdrawn.
\end{enumerate}


\section{General Meetings}\label{generalMeetings}
\subsection{Purpose}
A General Meeting of the club is the ultimate decision-making body of the club and has the power to direct the committee, the executive and the officers of the club. It is reserved for only important decisions which concern the club as a whole.

\subsection{Calling a meeting}
A General Meeting is considered ``called'' as long as the committee have reasonably made efforts to advertise the meeting to its members.

\subsection{Timing}
\begin{enumerate}[(1)]
    \item A president must call a meeting within 14 days when:
          \begin{enumerate}[(a)]
              \item the committee directs them to do as such; or
              \item they receive a petition calling for a GM from at least 7 club members.
          \end{enumerate}
    \item The meeting must be held no sooner than 10 working days after it is called.
    \item The meeting must be held not later than 1 calendar month after it is called.
\end{enumerate}

\subsection{Conducting a meeting}
The General Meeting is directed by the committee regarding whatever reason for which it was convened.

\subsection{Voting}
If a matter must be put to vote, then each member has 1 vote which they may not delegate to another. The quorum for a General Meeting in which a matter must be voted is 10 members.

\subsection{Record keeping}
The Secretary must record all the attending members in the minutes of the General Meeting.

\section{The Annual General Meeting}\label{annualGeneralMeeting}
\subsection{Definition}
The AGM is a special GM specifically for the purpose of conducting concluding business for the year. This includes reports from the President, Treasurer, and the Partnerships \& Sponsorships Manager, along with the election of a new committee. The process and rules are otherwise the same as a GM.

\subsection{Annual occurrence}
The AGM must be held during an academic day during the second half of the second semester every year.

\subsection{Returning Officer}
A Returning Officer must be appointed by the committee prior to the AGM. The returning officer cannot be a candidate for that election, but can otherwise be any willing member.

\subsection{Electing Officers}
A new committee must be formed at each AGM. The officers as stated in \Cref{committeeComp} must be elected by members of the club.

\section{The trustees upon winding up}
In the event that the Club winds up or lapses, the assets remaining, after the paying of debts and liabilities, shall be transferred to YouX, and those assets shall be used by YouX in accordance with YouX constitution.

\section{The Constitution}
\subsection{Interpretation}
This constitution is to be interpreted with close regard to the principles and objectives of the club as set out in \Cref{principlesObjectivesClub}.

\subsection{Effectiveness}
This constitution takes effect when it is ratified by club members in a GM.

\subsection{Amendments}
\begin{enumerate}[(1)]
    \item Amendments to this constitution may be proposed by a resolution of the committee or a petition from at least 10 members of the club. Any amendments to this constitution must be ratified by club members in a GM in order for the amendment to take effect.
    \item Minor typographical and formatting changes may be made to this constitution without the need for ratification by club members in a GM.
\end{enumerate}

\newpage
\textbf{THIS CONSTITUTION HAS BEEN REVIEWED AND RATIFIED BY CLUB MEMBERS IN A GENERAL MEETING.}

\vspace{0.5cm}
\textbf{DATE OF GENERAL MEETING:} 13th of March, 2024

\vspace{0.5cm}
\textbf{THIS CONSTITUTION HAS BEEN ENDORSED AND SIGNED BY}

\signature{David Maslov}{President}

\vspace{0.5cm}
\textbf{DATE OF ENDORSEMENT:} 13th of March, 2024


\end{document}
